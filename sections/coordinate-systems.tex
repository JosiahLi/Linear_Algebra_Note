\section{Coordinate Systems}
\subsection{Coordinates relative to a Basis}
\begin{Thm}
    \textbf{The Unique Representation Theorem}:
    Let $\mathcal{B} = \{\mathbf{b}_1, \cdots, \mathbf{b}_n \}$ be a basis for a vector space $V$. Then $\forall \mathbf{x}\in V$, there exists a unique set of scalars $c_1,\cdots, c_n$ such that
    \begin{equation*}
        \mathbf{x} = c_1\mathbf{b}_1 + \cdots  + c_n\mathbf{b}_n
    \end{equation*}
\end{Thm}
\begin{Def}
    $\mathcal{B}$-\textbf{coordinates}:
    Suppose $\mathcal{B} = \{\mathbf{b}_1, \cdots, \mathbf{b}_n \}$ is a basis for $V$ and $\mathbf{x}\in V$. The \textbf{coordinates of $\mathbf{x}$ relative to the basis $\mathcal{B}$ (or shortly coordinates of $\mathcal{B}$} are the weights $c_1,\cdots, c_n$ such that
    \begin{equation*}
        \mathbf{x} = c_1\mathbf{b}_1 + \cdots  + c_n\mathbf{b}_n
    \end{equation*}
    It is denoted by
    \begin{equation*}
        [\mathbf{x}]_{\mathcal{B}} = \begin{bmatrix}
            c_1\\
            \vdots\\
            c_n
        \end{bmatrix}
    \end{equation*}
    \begin{Rem}
        It is easy to see that $[\cdot]_{\mathcal{B}}$ is a linear transformation, that is:
        \begin{equation*}
            [c\mathbf{a} + \mathbf{b}]_\mathcal{B} = c[\mathbf{a}]_\mathcal{B} + [ \mathbf{b}]_\mathcal{B}
        \end{equation*}

        \noindent In fact, for any vector $\mathbf{x}$ in $\mathbb{R}^n$, its $\mathcal{E}$-coordinate is itself, where $\mathcal{E}$ is standard basis
    \end{Rem}
    

    \begin{equation*}
        [\mathbf{x}]_{\mathcal{E}} = \mathbf{x}
    \end{equation*}
    \end{Def}
    
    \subsection{Change of Coordiantes}
    \begin{Def}\label{4-1}
        \textbf{Change-of-Coordinates Matrix}: Let
        \begin{equation*}
            P_{\mathcal{B}} = \begin{bmatrix}
                \mathbf{b}_1 & \mathbf{b}_2 & \cdots & \mathbf{b}_n
            \end{bmatrix}
        \end{equation*}
        Then the vector equation $\mathbf{x} = c_1\mathbf{b}_1 + \cdots  + c_n\mathbf{b}_n$ is equivalent to
        \begin{equation*}
            \mathbf{x} = P_{\mathcal{B}}[\mathbf{x}]_{\mathcal{B}}
        \end{equation*}
        $P_{\mathcal{B}}$ is called \textbf{change-of-coordinates matrix} from $\mathbf{B}$ to \textbf{the standard basis} $\mathcal{E}$ in $\mathbb{R}^n$. Since $\mathcal{B}$ is a basis in $\mathbb{R}^n$, its inverse $P_{\mathcal{B}}^{-1}$ always exists. Left-multiplication by $P_{\mathcal{B}}^{-1}$ converts $\mathbf{x}$ into its $\mathcal{B}$-coordinate vector
        \begin{equation*}
            P_{\mathcal{B}}^{-1}\mathbf{x} = [\mathbf{x}]_{\mathcal{B}}
        \end{equation*}
    \end{Def}


\begin{Thm}
    Let $\mathcal{B} = \{\mathbf{b}_1, \cdots, \mathbf{b}_n\}$ and $\mathcal{C} = \{\mathbf{c}_1, \cdots, \mathbf{c}_n\}$ be bases of a vector space $V$. Then there is a unique $n\times n$ matrix $\underset{\mathcal{C} \leftarrow \mathcal{B}}{P}$ such that
    \begin{equation*}
        [\mathbf{x}]_{\mathcal{C}} = \underset{\mathcal{C} \leftarrow \mathcal{B}}{P}[\mathbf{x}]_{\mathcal{B}}
    \end{equation*}
    The columns of $\underset{\mathcal{C} \leftarrow \mathcal{B}}{P}$ are the $\mathcal{C}$-coordinate vectors of the vectors in the basis $\mathcal{B}$. That is,
    \begin{equation*}
        \underset{\mathcal{C} \leftarrow \mathcal{B}}{P} = \begin{bmatrix}
            [\mathbf{b}_1]_{\mathcal{C}} & [\mathbf{b}_2]_{\mathcal{C}} & \cdots & [\mathbf{b}_n]_{\mathcal{C}}
        \end{bmatrix}
    \end{equation*}
\end{Thm}
The matrix $\underset{\mathcal{C} \leftarrow \mathcal{B}}{P}$ is called \textbf{change-of-coordinates matrix from $\mathcal{B}$ to $\mathcal{C}$}.
 That is,
 \begin{equation*}
     [\mathbf{x}]_{\mathcal{C}} = \underset{\mathcal{C} \leftarrow \mathcal{B}}{P} [\mathbf{x}]_{\mathcal{B}}
 \end{equation*}
Similarly, the inverse of $\underset{\mathcal{C} \leftarrow \mathcal{B}}{P}$ always exists
\begin{equation*}
    (\underset{\mathcal{C} \leftarrow \mathcal{B}}{P})^{-1} = \underset{\mathcal{B} \leftarrow \mathcal{C}}{P}
\end{equation*}
Note that $P_{\mathcal{B}}$ implies that $\underset{\mathcal{E} \leftarrow \mathcal{B}}{P}$.
One of the ways to calculate $\underset{\mathcal{C} \leftarrow \mathcal{B}}{P}$ is to place the two sets of bases into a matrix, and then solve it as if it were a simple linear equation:
\begin{equation*}
    \begin{bmatrix}
        \mathcal{C} \mid  \mathcal{B} 
    \end{bmatrix}\sim [\ 
        I \mid \underset{\mathcal{C} \leftarrow \mathcal{B}}{P}\ 
    ]
\end{equation*}
