\section{Rank and Trace}
\subsection{Rank}
\begin{Def}
    The \textbf{rank} of a matrix $A\in \mathbb{R}^{n\times p}$ is the number of its linearly independent columns (or rows), which is expressed as $\text{rank}(A)$.
\end{Def}
\noindent 
Given a matrix $A\in \mathbb{R}^{n\times p}$, it has the following properties:
\begin{enumerate}
    \item $\text{rank}(A) = \min\{n, p \}$
    \item $\text{rank}(AB) = \min\{\text{rank}(A), \text{rank}(B) \}$
    \item Given two non-singular matrices $B\in\mathbb{R}^{n\times n}$ and $C\in\mathbb{R}^{p\times p}$:
    \begin{equation}
        \text{rank}(BA) = \text{rank}(AC) = \text{rank}(A)
    \end{equation}
    \item $\text{rank}(A^\top A) = \text{rank}(AA^{\top}) = \text{rank}(A) = \text{rank}(A^{\top})$
\end{enumerate}
Note that: property 4 illustrates that \textbf{multiplying A by a non-singular matrix does not change the rank of $A$}.

\begin{Ex}
    Show that if a matrix $A\in\mathbb{R}^{n\times p}$ with $n\geq p$ is of full column rank, then $A^{\top}A$ is non-singular.
\end{Ex}
\begin{proof}
    Since $A$ is of full column rank and $n\geq p$, we have
    \begin{equation*}
        \text{rank}(A) = p = \text{rank}(A^\top A)
    \end{equation*}
    Since $A^\top A$ is a $p\times p$ matrix and has full column rank, it is non-singular.
\end{proof}

\begin{Ex}
    Show that if a matrix $A\in\mathbb{R}^{n\times p}$ with $n\geq p$ is not of full column rank, then $A^{\top}A$ is singular.
\begin{proof}
    Since $A$ is not of full column rank, 
    \begin{equation*}
        \text{rank}(A) = \text{rank}(A^{\top} A) < p
    \end{equation*}
    It implies that $A$ is singular.
\end{proof}
\end{Ex}

\begin{Ex}
    Show that given a matrix $A\in\mathbb{R}^{n\times p}$ with $n < p$, $A^{\top}A$ is singular.
    \begin{proof}
        Since $\text{rank}(A) \leq \min\{n, p \}$,
        \begin{equation*}
            \text{rank}(A) = \text{rank}(A^{\top} A) \leq n < p
        \end{equation*}
        Since $A^\top A$ is not of full column rank, it is singular.
    \end{proof}
\end{Ex}

\subsection{Trace}
\begin{Def}
    The trace of a square matrix $A\in \R^{n \times n}$ is the sum of diagonal elements of $A$. It is denoted $\tr(A) = \sum_{i = 1}^n = a_{ii}$.
\end{Def}

\begin{Thm}\label{trace}
    The trace function $\tr(\cdot)$ has the following properties:
    \begin{enumerate}
        \item $\tr(cA \pm dB) = c\tr(A) \pm d\tr(B)$, where $c, d\in \R$.
        \item Given two matrices $A\in\R^{n\times p}, B\in \R^{p\times n}$, then $\tr(AB) = \tr(BA)$
        \begin{proof}
            Let $t_i$ be the $i^{\text{th}}$ elements on the diagonal of $AB$. Then
            \begin{equation*}
                \tr(AB) = \sum_{i = 1}^n t_i = \sum_{i = 1}^n\sum_{j = 1}^p a_{ij}b_{ji} = \sum_{j = 1}^p \sum_{i = 1}^n b_{ji}a_{ij} = \tr(BA)
            \end{equation*}
            Note that $n$ is not required to be greater or equal to $p$.
        \end{proof}

    \item Given an $n\times p$ matrix, $A = \begin{bmatrix}
        \B{a}_1 & \B{a}_2 & \cdots & \B{a}_p
    \end{bmatrix}$, $\tr(\T{A}A) = \displaystyle \sum_{i = 1}^p \T{\B{a}}_i \B{a}_i$
    \item Given an $n \times p$ matrix, $\tr(A\T{A}) = \displaystyle \sum_{i = 1}^ n \B{a}^{(i)} \B{a}_i$, where $\B{a}^{(1)}$ is the row vector of $A$.
    \item By property 3 and 4, $\tr(\T{A}A) = \tr(A\T{A}) = \displaystyle \sum_{i = 1}^n\sum_{j = 1}^p a_{ij}^2$
    \item $\tr(\E(\B{X})) = \E(\tr(\B{X}))$, where $\E$ represents the expectation of a random matrix.
    \end{enumerate}
\end{Thm}